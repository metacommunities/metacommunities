%% ================================================================================
%% This LaTeX file was created by AbiWord.                                         
%% AbiWord is a free, Open Source word processor.                                  
%% More information about AbiWord is available at http://www.abisource.com/        
%% ================================================================================

\documentclass[a4,portrait,11pt]{article}
\usepackage[latin1]{inputenc}
\usepackage{calc}
\usepackage{setspace}
\usepackage{fixltx2e}
\usepackage{graphicx}
\usepackage{multicol}
\usepackage[normalem]{ulem}
%% Please revise the following command, if your babel
%% package does not support en-GB
\usepackage[en]{babel}
\usepackage{color}
\usepackage{hyperref}
 
\begin{document}

\setlength{\oddsidemargin}{0.7875in-1in}
\setlength{\textwidth}{\paperwidth - 0.7875in-0.7875in}

\begin{flushleft}
\textbf{Infrastructures and Social Complexity: A Routledge Companion }
\end{flushleft}


\begin{flushleft}
\textbf{Edited by Penny Harvey, Casper Bruun Jensen \& Atsuro Morita}
\end{flushleft}


\begin{flushleft}

\end{flushleft}


\begin{flushleft}
Infrastructures are conventionally conceived as large-scale material systems, like roads or sewage systems, that support societal aims or public goods. In recent years, however, this traditional and technology-oriented view has been opened up in consequence of new empirical developments as well as novel forms of conceptual engagement. 
\end{flushleft}


\begin{flushleft}

\end{flushleft}


\begin{flushleft}
Empirically, infrastructures have expanded with the advent and explosive spread of social media, platforms for knowledge sharing, and a wide range of sensory materials, smart devices and surveillance systems. These information infrastructures are now routinely integrated with classical infrastructures, as in real-time monitoring systems of social and natural occurrences, from tsunami or flood prediction to intelligent electricity grids or traffic flow systems. The impact of digital technologies on the material fabric of social systems is not the only source of the new-found visibility of the infrastructural. While infrastructures were never {`}invisible' to those who worked to implement, maintain or contest them (as scholars of infrastructural innovation have demonstrated in a number of classic studies), the contemporary mode of public awareness of risk and uncertainty has put infrastructures firmly onto the social agenda. Public authorities are increasingly concerned with the critical limits of infrastructural support both at the scale of planetary ecological systems and human lifeworlds, and in relation to the specific constituencies to whom they are accountable. Concerns over energy, water and food security drive public agendas in response to the uncertainties provoked by climate change and environmental degradation. At the same time at the level of political economy, the widespread adoption of neo-liberal policies that seek to reduce state control (both financial and regulatory) of infrastructural systems has seen a huge growth in public-private finance initiatives that disrupt previous notions of infrastructures as public systems, and raise new challenges in the face of distributed ownership and responsibility. Furthermore, the growing importance of infrastructures as capital assets in their own right, and the enhanced visibility of trans-national investment raises new challenges for scholars who seek to analyse the social processes through which infrastructures are produced and maintained (or not). 
\end{flushleft}


\begin{flushleft}

\end{flushleft}


\begin{flushleft}
The increased empirical visibility of infrastructures as sites of social concern is echoed by the potential that these systems offer for conceptual innovation. Scholarship from science and technology studies, anthropology, urban geography, media studies and continental philosophy have challenged the dualist foundations upon which the conventional understanding of infrastructure rests. Shared among these scholars, is a view of infrastructures as complex, dynamic and fragile assemblages. These approaches challenge the conventional analytical distinctions between the technical and the socio-political, the natural and cultural, the material and the social. They show that infrastructural work is actively reconfiguring these very domains in different ways and at different scales. 
\end{flushleft}


\begin{flushleft}

\end{flushleft}


\begin{flushleft}
This book draws these concerns together by gathering some of the most influential international scholarship in this emergent field around six core thematic areas: environmental, data, energy, economic, media/communications and civic (and/or) urban -- infrastructures. Our aim is to produce an overview of infrastructural research that works across thematic fields, and across disciplinary divides in ways that will offer both an orientation to researchers and students new to the field, and to begin the work of giving shape to a dispersed field as it is currently taking form. We would hope that this volume would become a benchmark text for MA, PhDs and established researchers in anthropology, sociology, STS and urban geography. 
\end{flushleft}


\begin{flushleft}
\textbf{\newpage
The core questions that we would like you to address in your contribution:}
\end{flushleft}


\begin{flushleft}

\end{flushleft}


\begin{flushleft}

\end{flushleft}


\begin{flushleft}
How do you methodologically engage and analytically conceive of infrastructure. What conceptual genealogies do you draw on?
\end{flushleft}


\begin{flushleft}

\end{flushleft}


\begin{flushleft}
Empirically how are you specifying the infrastructural: what is an infrastructure, for whom and in relation to what specific practices?   
\end{flushleft}


\begin{flushleft}

\end{flushleft}


\begin{flushleft}
What constitutes the specific infrastructural assemblage with which you are concerned? What do infrastructures hold together (materials, institutions, practices, discourses, etc.) What is the {`}stuff' of a specific infrastructural arrangement?
\end{flushleft}


\begin{flushleft}

\end{flushleft}


\begin{flushleft}
How are things held together (practices, techniques, relationships, standards, norms)? What are the modes of translation and/or conversion that a specific infrastructural form/dynamic implies?
\end{flushleft}


\begin{flushleft}

\end{flushleft}


\begin{flushleft}
How do you engage the emergent and unforeseeable qualities of infrastructure, what we might call their experimental qualities?
\end{flushleft}


\begin{flushleft}

\end{flushleft}


\begin{flushleft}

\end{flushleft}


\begin{flushleft}
Sub questions might include: 
\end{flushleft}


\begin{flushleft}

\end{flushleft}


\begin{flushleft}
*How do infrastructures shape futures? What are the temporalities of the infrastructure? 
\end{flushleft}


\begin{flushleft}

\end{flushleft}


\begin{flushleft}
*How do infrastructures shape new worlds or ontologies.  
\end{flushleft}


\begin{flushleft}

\end{flushleft}


\begin{flushleft}
*How do infrastructures make space and create new scales of social interaction
\end{flushleft}


\begin{flushleft}

\end{flushleft}


\begin{flushleft}
*What are the limits of infrastructure (where do they end, for whom, how and why)?
\end{flushleft}


\begin{flushleft}

\end{flushleft}


\begin{flushleft}
*What are the materialities of infrastructure and what is their relation to {`}ideals' -- like discourses, visions, plans.
\end{flushleft}


\begin{flushleft}

\end{flushleft}


\begin{flushleft}
*What is the relation between emergent infrastructures, new forms of politics and forms of social change, including resistance.
\end{flushleft}


\begin{flushleft}

\end{flushleft}


\begin{flushleft}

\end{flushleft}


\begin{flushleft}

\end{flushleft}


\begin{flushleft}
\newpage

\end{flushleft}








\end{document}
